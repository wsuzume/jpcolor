%Japanese Traditional Color for LaTeX
%Author: syo
%Date: 2020.01.02

\definecolor{nadeshiko}{RGB}{220,159,180} %撫子
\definecolor{kohbai}{RGB}{225,107,140} %紅梅
\definecolor{suoh}{RGB}{142,53,74} %蘇芳
\definecolor{taikoh}{RGB}{248,195,205} %退紅
\definecolor{ikkonzome}{RGB}{244,167,185} %一斥染
\definecolor{kuwazome}{RGB}{100,54,60} %桑染
\definecolor{momo}{RGB}{245,150,170} %桃
\definecolor{ichigo}{RGB}{181,73,91} %苺
\definecolor{usubeni}{RGB}{232,122,144} %薄紅
\definecolor{imayoh}{RGB}{208,90,110} %今様
\definecolor{nakabeni}{RGB}{219,77,109} %中紅
\definecolor{sakura}{RGB}{254,223,225} %桜
\definecolor{umenezumi}{RGB}{158,122,122} %梅鼠
\definecolor{karakurenai}{RGB}{208,16,76} %韓紅花
\definecolor{enji}{RGB}{159,53,58} %燕脂
\definecolor{kurenai}{RGB}{203,27,69} %紅
\definecolor{toki}{RGB}{238,169,169} %鴇
\definecolor{cyohsyun}{RGB}{191,103,102} %長春
\definecolor{kokiake}{RGB}{134,71,63} %深緋
\definecolor{sakuranezumi}{RGB}{177,150,147} %桜鼠
\definecolor{jinzamomi}{RGB}{235,122,119} %甚三紅
\definecolor{azuki}{RGB}{149,74,69} %小豆
\definecolor{suohkoh}{RGB}{169,99,96} %蘇芳香
\definecolor{akabeni}{RGB}{203,64,66} %赤紅
\definecolor{shinsyu}{RGB}{171,59,58} %真朱
\definecolor{haizakura}{RGB}{215,196,187} %灰桜
\definecolor{kuriume}{RGB}{144,72,64} %栗梅
\definecolor{ebicha}{RGB}{115,67,56} %海老茶
\definecolor{ginsyu}{RGB}{199,62,58} %銀朱
\definecolor{kurotobi}{RGB}{85,66,54} %黒鳶
\definecolor{benitobi}{RGB}{153,70,57} %紅鳶
\definecolor{akebono}{RGB}{241,148,131} %曙
\definecolor{benikaba}{RGB}{181,68,52} %紅樺
\definecolor{mizugaki}{RGB}{185,136,125} %水がき
\definecolor{sangosyu}{RGB}{241,124,103} %珊瑚朱
\definecolor{benihiwada}{RGB}{136,76,58} %紅檜皮
\definecolor{syojyohi}{RGB}{232,48,21} %猩猩緋
\definecolor{entan}{RGB}{215,84,85} %鉛丹
\definecolor{shikancha}{RGB}{181,93,76} %芝翫茶
\definecolor{hiwada}{RGB}{133,72,54} %檜皮
\definecolor{kakishibu}{RGB}{163,94,71} %柿渋
\definecolor{ake}{RGB}{204,84,58} %緋
\definecolor{tobi}{RGB}{114,72,50} %鳶
\definecolor{benihi}{RGB}{247,92,47} %紅緋
\definecolor{kurikawacha}{RGB}{106,64,40} %栗皮茶
\definecolor{bengara}{RGB}{154,80,52} %弁柄
\definecolor{terigaki}{RGB}{196,98,67} %照柿
\definecolor{edocha}{RGB}{175,95,60} %江戸茶
\definecolor{araisyu}{RGB}{251,150,110} %洗朱
\definecolor{momoshiocha}{RGB}{114,73,56} %百塩茶
\definecolor{karacha}{RGB}{180,113,87} %唐茶
\definecolor{tokigaracha}{RGB}{219,142,113} %ときがら茶
\definecolor{ohni}{RGB}{240,94,28} %黄丹
\definecolor{sohi}{RGB}{237,120,74} %纁
\definecolor{ensyucha}{RGB}{202,120,83} %遠州茶
\definecolor{kabacha}{RGB}{179,92,55} %樺茶
\definecolor{kogecha}{RGB}{86,63,46} %焦茶
\definecolor{akakoh}{RGB}{227,145,110} %赤香
\definecolor{suzumecha}{RGB}{143,90,60} %雀茶
\definecolor{shishi}{RGB}{240,169,134} %宍
\definecolor{sodenkaracha}{RGB}{160,103,75} %宗伝唐茶
\definecolor{kaba}{RGB}{193,105,60} %樺
\definecolor{kokikuchinashi}{RGB}{251,153,102} %深支子
\definecolor{kurumi}{RGB}{148,122,109} %胡桃
\definecolor{taisya}{RGB}{163,99,54} %代赭
\definecolor{araigaki}{RGB}{231,148,96} %洗柿
\definecolor{kohrozen}{RGB}{125,83,44} %黄櫨染
\definecolor{akakuchiba}{RGB}{199,133,80} %赤朽葉
\definecolor{tonocha}{RGB}{152,95,42} %礪茶
\definecolor{akashirotsurubami}{RGB}{225,166,121} %赤白橡
\definecolor{sencha}{RGB}{133,91,50} %煎茶
\definecolor{kanzo}{RGB}{252,159,77} %萱草
\definecolor{sharegaki}{RGB}{255,186,132} %洒落柿
\definecolor{beniukon}{RGB}{233,139,42} %紅鬱金
\definecolor{umezome}{RGB}{233,163,104} %梅染
\definecolor{biwacha}{RGB}{177,120,68} %枇杷茶
\definecolor{chojicha}{RGB}{150,99,46} %丁子茶
\definecolor{kenpohzome}{RGB}{67,52,27} %憲法染
\definecolor{kohaku}{RGB}{202,122,44} %琥珀
\definecolor{usugaki}{RGB}{236,184,138} %薄柿
\definecolor{kyara}{RGB}{120,85,43} %伽羅
\definecolor{chojizome}{RGB}{176,119,54} %丁子染
\definecolor{fushizome}{RGB}{150,114,73} %柴染
\definecolor{kuchiba}{RGB}{226,148,59} %朽葉
\definecolor{kincha}{RGB}{199,128,45} %金茶
\definecolor{kitsune}{RGB}{155,110,35} %狐
\definecolor{susutake}{RGB}{110,85,47} %煤竹
\definecolor{usukoh}{RGB}{235,180,113} %薄香
\definecolor{tonoko}{RGB}{215,185,142} %砥粉
\definecolor{ginsusutake}{RGB}{130,102,58} %銀煤竹
\definecolor{ohdo}{RGB}{182,142,85} %黄土
\definecolor{shiracha}{RGB}{188,159,119} %白茶
\definecolor{kobicha}{RGB}{135,102,51} %媚茶
\definecolor{kigaracha}{RGB}{193,138,38} %黄唐茶
\definecolor{yamabuki}{RGB}{255,177,27} %山吹
\definecolor{yamabukicha}{RGB}{209,152,38} %山吹茶
\definecolor{hajizome}{RGB}{221,165,45} %櫨染
\definecolor{kuwacha}{RGB}{201,152,51} %桑茶
\definecolor{tamago}{RGB}{249,191,69} %玉子
\definecolor{shirotsurubami}{RGB}{220,184,121} %白橡
\definecolor{kitsurubami}{RGB}{186,145,50} %黄橡
\definecolor{tamamorokoshi}{RGB}{232,182,71} %玉蜀黍
\definecolor{hanaba}{RGB}{247,194,66} %花葉
\definecolor{namakabe}{RGB}{125,108,70} %生壁
\definecolor{torinoko}{RGB}{218,201,166} %鳥の子
\definecolor{usuki}{RGB}{250,214,137} %浅黄
\definecolor{kikuchiba}{RGB}{217,171,66} %黄朽葉
\definecolor{kuchinashi}{RGB}{246,197,85} %梔子
\definecolor{tohoh}{RGB}{255,196,8} %籐黄
\definecolor{ukon}{RGB}{239,187,36} %鬱金
\definecolor{karashi}{RGB}{202,173,95} %芥子
\definecolor{higosusutake}{RGB}{141,116,42} %肥後煤竹
\definecolor{rikyushiracha}{RGB}{180,165,130} %利休白茶
\definecolor{aku}{RGB}{135,127,108} %灰汁
\definecolor{rikyucha}{RGB}{137,125,85} %利休茶
\definecolor{rokohcha}{RGB}{116,103,62} %路考茶
\definecolor{nataneyu}{RGB}{162,140,55} %菜種油
\definecolor{uguisucha}{RGB}{108,96,36} %鶯茶
\definecolor{kimirucha}{RGB}{134,120,53} %黄海松茶
\definecolor{mirucha}{RGB}{98,89,44} %海松茶
\definecolor{kariyasu}{RGB}{233,205,76} %刈安
\definecolor{nanohana}{RGB}{247,217,76} %菜の花
\definecolor{kihada}{RGB}{251,226,81} %黄蘗
\definecolor{mushikuri}{RGB}{217,205,144} %蒸栗
\definecolor{aokuchiba}{RGB}{173,161,66} %青朽葉
\definecolor{ominaeshi}{RGB}{221,210,59} %女郎花
\definecolor{hiwacha}{RGB}{165,160,81} %鶸茶
\definecolor{hiwa}{RGB}{190,194,63} %鶸
\definecolor{uguisu}{RGB}{108,106,45} %鶯
\definecolor{yanagicha}{RGB}{147,150,80} %柳茶
\definecolor{koke}{RGB}{131,138,45} %苔
\definecolor{kikujin}{RGB}{177,180,121} %麹塵
\definecolor{rikancha}{RGB}{97,97,56} %璃寛茶
\definecolor{aikobicha}{RGB}{75,78,42} %藍媚茶
\definecolor{miru}{RGB}{91,98,46} %海松
\definecolor{sensaicha}{RGB}{77,81,57} %千歳茶
\definecolor{baikocha}{RGB}{137,145,107} %梅幸茶
\definecolor{hiwamoegi}{RGB}{144,180,75} %鶸萌黄
\definecolor{yanagizome}{RGB}{145,173,112} %柳染
\definecolor{urayanagi}{RGB}{181,202,160} %裏柳
\definecolor{iwaicha}{RGB}{100,106,88} %岩井茶
\definecolor{moegi}{RGB}{123,162,63} %萌黄
\definecolor{nae}{RGB}{134,193,102} %苗
\definecolor{yanagisusutake}{RGB}{74,89,61} %柳煤竹
\definecolor{matsuba}{RGB}{66,96,45} %松葉
\definecolor{aoni}{RGB}{81,110,65} %青丹
\definecolor{usuao}{RGB}{145,180,147} %薄青
\definecolor{yanaginezumi}{RGB}{128,143,124} %柳鼠
\definecolor{tokiwa}{RGB}{27,129,62} %常磐
\definecolor{wakatake}{RGB}{93,172,129} %若竹
\definecolor{chitosemidori}{RGB}{54,86,60} %千歳緑
\definecolor{midori}{RGB}{34,125,81} %緑
\definecolor{byakuroku}{RGB}{168,216,185} %白緑
\definecolor{oitake}{RGB}{106,131,114} %老竹
\definecolor{tokusa}{RGB}{45,109,75} %木賊
\definecolor{onandocha}{RGB}{70,93,76} %御納戸茶
\definecolor{rokusyoh}{RGB}{36,147,110} %緑青
\definecolor{sabiseiji}{RGB}{134,166,151} %錆青磁
\definecolor{aotake}{RGB}{0,137,108} %青竹
\definecolor{veludo}{RGB}{9,97,72} %ビロード
\definecolor{mushiao}{RGB}{32,96,79} %虫襖
\definecolor{aimirucha}{RGB}{15,76,58} %藍海松茶
\definecolor{tonocha2}{RGB}{79,114,108} %沈香茶
\definecolor{aomidori}{RGB}{0,170,144} %青緑
\definecolor{seiji}{RGB}{105,176,172} %青磁
\definecolor{tetsu}{RGB}{38,69,61} %鉄
\definecolor{mizuasagi}{RGB}{102,186,183} %水浅葱
\definecolor{seiheki}{RGB}{38,135,133} %青碧
\definecolor{sabitetsuonando}{RGB}{64,91,85} %錆鉄御納戸
\definecolor{korainando}{RGB}{48,90,86} %高麗納戸
\definecolor{byakugun}{RGB}{120,194,196} %白群
\definecolor{omeshicha}{RGB}{55,107,109} %御召茶
\definecolor{kamenozoki}{RGB}{165,222,228} %瓶覗
\definecolor{fukagawanezumi}{RGB}{119,150,154} %深川鼠
\definecolor{sabiasagi}{RGB}{102,153,161} %錆浅葱
\definecolor{mizu}{RGB}{129,199,212} %水
\definecolor{asagi}{RGB}{51,166,184} %浅葱
\definecolor{onando}{RGB}{12,72,66} %御納戸
\definecolor{ai}{RGB}{13,86,97} %藍
\definecolor{shinbashi}{RGB}{0,137,167} %新橋
\definecolor{sabionando}{RGB}{51,103,116} %錆御納戸
\definecolor{tetsuonando}{RGB}{37,83,89} %鉄御納戸
\definecolor{hanaasagi}{RGB}{30,136,168} %花浅葱
\definecolor{ainezumi}{RGB}{86,108,115} %藍鼠
\definecolor{masuhana}{RGB}{87,124,138} %舛花
\definecolor{sora}{RGB}{88,178,220} %空
\definecolor{noshimehana}{RGB}{43,95,117} %熨斗目花
\definecolor{chigusa}{RGB}{58,143,183} %千草
\definecolor{omeshionando}{RGB}{46,92,110} %御召御納戸
\definecolor{hanada}{RGB}{0,98,132} %縹
\definecolor{wasurenagusa}{RGB}{125,185,222} %勿忘草
\definecolor{gunjyo}{RGB}{81,168,221} %群青
\definecolor{tsuyukusa}{RGB}{46,169,223} %露草
\definecolor{kurotsurubami}{RGB}{11,16,19} %黒橡
\definecolor{kon}{RGB}{15,37,64} %紺
\definecolor{kachi}{RGB}{8,25,45} %褐
\definecolor{ruri}{RGB}{0,92,175} %瑠璃
\definecolor{rurikon}{RGB}{11,52,110} %瑠璃紺
\definecolor{benimidori}{RGB}{123,144,210} %紅碧
\definecolor{fujinezumi}{RGB}{110,117,164} %藤鼠
\definecolor{tetsukon}{RGB}{38,30,71} %鉄紺
\definecolor{konjyo}{RGB}{17,50,133} %紺青
\definecolor{benikakehana}{RGB}{78,79,151} %紅掛花
\definecolor{konkikyo}{RGB}{33,30,85} %紺桔梗
\definecolor{fuji}{RGB}{139,129,195} %藤
\definecolor{futaai}{RGB}{112,100,154} %二藍
\definecolor{ouchi}{RGB}{155,144,194} %楝
\definecolor{fujimurasaki}{RGB}{138,107,190} %藤紫
\definecolor{kikyo}{RGB}{106,76,156} %桔梗
\definecolor{shion}{RGB}{143,119,181} %紫苑
\definecolor{messhi}{RGB}{83,61,91} %滅紫
\definecolor{usu}{RGB}{178,143,206} %薄
\definecolor{hashita}{RGB}{152,109,178} %半
\definecolor{edomurasaki}{RGB}{119,66,141} %江戸紫
\definecolor{shikon}{RGB}{60,47,65} %紫紺
\definecolor{kokimurasaki}{RGB}{74,34,93} %深紫
\definecolor{sumire}{RGB}{102,50,124} %菫
\definecolor{murasaki}{RGB}{89,44,99} %紫
\definecolor{ayame}{RGB}{111,51,129} %菖蒲
\definecolor{fujisusutake}{RGB}{87,76,87} %藤煤竹
\definecolor{benifuji}{RGB}{180,129,187} %紅藤
\definecolor{kurobeni}{RGB}{63,43,54} %黒紅
\definecolor{nasukon}{RGB}{87,42,63} %茄子紺
\definecolor{budohnezumi}{RGB}{94,61,80} %葡萄鼠
\definecolor{hatobanezumi}{RGB}{114,99,110} %鳩羽鼠
\definecolor{kakitsubata}{RGB}{98,41,84} %杜若
\definecolor{ebizome}{RGB}{109,46,91} %蒲葡
\definecolor{botan}{RGB}{193,50,142} %牡丹
\definecolor{umemurasaki}{RGB}{168,73,122} %梅紫
\definecolor{nisemurasaki}{RGB}{86,46,55} %似紫
\definecolor{tsutsuji}{RGB}{224,60,138} %躑躅
\definecolor{murasakitobi}{RGB}{96,55,62} %紫鳶
\definecolor{shironeri}{RGB}{252,250,242} %白練
\definecolor{gofun}{RGB}{255,255,251} %胡粉
\definecolor{shironezumi}{RGB}{189,192,186} %白鼠
\definecolor{ginnezumi}{RGB}{145,152,159} %銀鼠
\definecolor{namari}{RGB}{120,120,120} %鉛
\definecolor{hai}{RGB}{130,130,130} %灰
\definecolor{sunezumi}{RGB}{120,125,123} %素鼠
\definecolor{rikyunezumi}{RGB}{112,124,116} %利休鼠
\definecolor{nibi}{RGB}{101,103,101} %鈍
\definecolor{aonibi}{RGB}{83,89,83} %青鈍
\definecolor{dobunezumi}{RGB}{79,79,72} %溝鼠
\definecolor{benikeshinezumi}{RGB}{82,67,61} %紅消鼠
\definecolor{aisumicha}{RGB}{55,60,56} %藍墨茶
\definecolor{binrojizome}{RGB}{58,50,38} %檳榔子染
\definecolor{keshizumi}{RGB}{67,67,67} %消炭
\definecolor{sumi}{RGB}{28,28,28} %墨
\definecolor{kuro}{RGB}{8,8,8} %黒
\definecolor{ro}{RGB}{12,12,12} %呂
